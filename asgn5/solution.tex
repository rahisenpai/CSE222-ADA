\documentclass{article}
\usepackage[utf8]{inputenc}
\usepackage{amsmath}
\usepackage{algorithm}
\usepackage{algpseudocode}

\usepackage[margin=1.5cm]{geometry}

\title{Theory Assignment-5: ADA Winter-2024}
\author{Himanshu Raj (2022216) \and Tanish Verma (2022532)}

\date{April 21, 2024}
\begin{document}

\maketitle

\section{Flow Network construction}
% Your preprocessing steps go here
Add two additional vertices s and t, where s is a source vertex and t is a sink vertex. Let each box b$_i$ be represented by two vertices u$_i$ and v$_i$.\\
\textbf{All edges are directed and have capacity 1}. Add an edge from s to u$_i$ for all i. Add an edge from v$_i$ to t for all i.\\
Add an edge from u$_i$ to v$_j$, iff v$_j$(b$_j$) can be nested into u$_i$(b$_i$). Note that the box b$_j$ fits into b$_i$ iff dimensions of b$_j$ are strictly smaller than dimensions of b$_i$ in a specific configuration (or rotation) as stated in the problem itself.

\section{Claims}
% Describe your algorithm here
Claim 1: For u$_i$ for all i, there is at most one outgoing edge having flow 1.\\
Reasoning: At most one unit of inflow is possible from s for any u$_i$.\\
\\
Claim 2: For v$_i$ for all i, there is at most one incoming edge having flow 1.\\
Reasoning: At most one unit of outflow is possible from any v$_i$ to t.\\
\\
For Claims 3 and 4, since they deal with maximum flow and edges cases, let us discuss a Case which we'll refer to as Case 1 below.\\
Case 1:  Let's suppose there exists a set of boxes where every box can be nested into some box except the biggest one. Let the smallest box be b$_1$, the second smallest box be b$_2$, and so on, the largest box b$_n$.\\
\\
Claim 3: The maximum flow can be at most n-1, where n is the number of boxes.\\
Reasoning: Let the smallest box be b$_k$, in Case 1. According to our construction, no incoming edge exists for v$_k$, and hence there will be no outflow from v$_k$. \\
Given n boxes, each box can have maximum flow 1 as per Claim 1 and Claim 2. Now, if for each box b$_i$ in this case which is not b$_k$, there is a maximum flow 1 through u$_i$ .\\
For the maximum network flow, there is a case where each u$_i$ is connected to at least one v$_i$ having a valid flow from u$_i$ to v$_i$. (Refer the Case 1).
Since, all boxes from b$_2$, b$_3$ ... b$_n$ (but not the smallest box, b$_1$ in this case) , have a valid flow of 1, the maximum flow is n - 1.\\
\\
Claim 4: The maximum number of edges can be at most (n$^2$+3n)/2, where n is the number of boxes.\\
Reasoning: In Case 1,\\ 
For b$_n$, the maximum number of boxes which can be nested inside it can be n-1 in the worst case, that is, n-1 edges from u$_n$ to v$_1$, v$_2$ ... v$_{n-1}$.\\For b$_{n-1}$ this maximum number of edges is n-2 , that is, from u$_{n-1}$ to  v$_1$, v$_2$ ... v$_{n-1}$.
Therefore, for b$_i$ , this number is n - (i+1).\\
The summation of edges becomes (n - 1) + (n - 2) + (n - 3) + ... + 2 + 1 which reduces to n*(n-1)/2.\\
Also, there are n edges from the source s to all u$_i$, and another n edges from all v$_i$ to sink t.\\
Therefore, maximum number of edges possible is 2*n + ((n*(n-1)/2) which is (n$^2$ + 3n)/2.\\
\\ 
Claim 5: Minimum Number of Visible Boxes = Number of Boxes - Max Flow from s to t.\\
Reasoning: This is because the Max Flow from s to t gives the total number of boxes that can be nested. Since each flow of value 1 from u$_i$ to v$_j$ represents one nesting, and each nesting can be seen as reduction of one box from the total number of boxes.\\

\section{Proof of Correctness}
% Provide the proof of correctness for your algorithm
The constructed network is a maximum matching problem in bipartite graph (from A to B where A represents all u$_i$ and B represents all v$_j$) .\\
After Flow Network construction, run the Ford-Fulkerson algorithm to find the maximum flow from s to t.\\
The Ford-Fulkerson algorithm gives a maximum-bipartite matching. This means that for every u$_i$ matched with some v$_j$, v$_j$ can be nested in u$_i$. This matching ensures that each box is nested inside only one other box at max\\
Finally, the max flow in this maximum matching is the maximum number of times one box can be nested into a bigger one. Since every time a flow from s to t exists (during the execution of Ford-Fulkerson), the number of visible boxes reduces by 1.\\
\\
Therefore, Minimum Number of Visible Boxes = Number of Boxes - Max Flow from s to t\\

\section{Runtime Analysis}
% Analyze the time and space complexity of your algorithm
Let n be the total number of boxes.
For the flow network construction, $|$V$|$ = 2*n + 2\\
and the maximum number of edges can be of order n$^2$ at the worst case. The maximum number of edges in a directed graph can be at most $|$V$|$*($|$V$|$ - 1), but here this number is even less -$>$ $|$E$|$ = 2*n + n(n-1)/2 (following from Claim 4)\\
Therefore, the flow network construction takes O($|$V$|$ + $|$E$|$) time in adjacency list representation, which is simply O(n$^2$).\\
\\
The Ford-Fulkerson algorithm used to check maximum matching takes O(value(flow)($|$V$|$ + $|$E$|$)-time.\\
The value(flow) can be n-1 in the worst case. Therefore, this takes O(n-1)*O(n$^2$) time, which is O(n$^3$) time in the worst case.

\end{document}